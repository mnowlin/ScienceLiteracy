\documentclass{beamer}
%\usepackage{beamerthemeHannover}
\usepackage[round]{natbib}
\usepackage{amsmath}
\setbeamercovered{invisible}
\usepackage[mathletters]{ucs}
\usepackage[utf8x]{inputenc}
%% \setlength{\parindent}{0pt}
%% \setlength{\parskip}{6pt plus 2pt minus 1pt}
%% \setcounter{secnumdepth}{0}

\title{Ideology, Science Literacy, and the Polarization of
Science-Based Policy Issues}
\author{Matthew C. Nowlin}
\institute
{Department of Political Science \\
College of Charleston \\
\url{nowlinmc@cofc.edu} \\ 
\url{https://github.com/mnowlin/ScienceLiteracy}}
\date{October 2014}


% remove navigation bar
\setbeamertemplate{navigation symbols}{}

% transparent overlays
% \setbeamercovered{transparent}

% for natbib
\def\newblock{}

\begin{document}

\frame {\titlepage}

\begin{frame}\frametitle{Research Question(s)}

\begin{itemize}
\itemsep1pt\parskip0pt\parsep0pt
\item
  Does increased science literacy decrease polarization on
  science--based policy issues? \vspace{0.25in}
\item
  How is science literacy moderated by ideological and partisan
  attachments? \vspace{0.25in}
\item
  Is there a difference in impacts based on \emph{general} scientific
  knowledge vs. \emph{issue specific} knowledge?
\end{itemize}

\end{frame}

\begin{frame}\frametitle{Science Literacy}

\begin{itemize}
\itemsep1pt\parskip0pt\parsep0pt
\item
  Some argue that polarization on science issues result from a lack of
  basic understanding of science, or from hostility to science from
  certain groups. Indeed, previous has some, on issue like climate
  change, knowledge tends to lead toward convergence of opinion
  \vspace{0.25in}
\item
  \emph{H1: Increased science literacy will decrease issue polarization}
\end{itemize}

\end{frame}

\begin{frame}{Motivated Reasoning and Sophistication}

\begin{itemize}
\item 
However, others argue that scientific information is processed in a
“motivated” way, where information that confirms previous bias is
accepted and information that is counter to bias is rejected. Going
further, there is evidence to suggest that those with higher levels of
sophistication (e.g., education, political knowledge, science
literacy) tend to be those most polarized.
\vspace{0.25in}
    
\item 
 \textit{H2: Increased science literacy will increase issue
   polarization}
\end{itemize}

\end{frame}

\begin{frame}{Data and Methods}
  \begin{itemize}
  \item The data used to test these hypotheses are from a \textit{Pew
      Research Center for the People and the Press} survey
    administered from April 28 to May 12, 2009. The RDD telephone
    survey was given to a sample of 2,001 adults in the U.S. through
    both landlines and cell phones.   
  \end{itemize}  
\end{frame}

\begin{frame}{Independent Variables}

\begin{center}
{
\begin{tabular}{lrrrr}
 \textbf{Variable} & $\mathbf{\bar{X}}$ & \textbf{Min} & \textbf{Max} & $\mathbf{sd}$ \\ 
  \hline
Age & 51.37 & 18 & 95 & 17.58 \\ 
  Male &  0.50 &  0 &  1 &  0.50 \\ 
  White &  0.79 &  0 &  1 &  0.41 \\ 
  Education &  4.62 &  1 &  7 &  1.69 \\ 
  Income &  5.21 &  1 &  9 &  2.36 \\ 
  Born Again &  0.44 &  0 &  1 &  0.50 \\ 
  Church Attendance &  2.68 &  0 &  5 &  1.59 \\ 
  Liberal &  0.21 &  0 &  1 &  0.40 \\ 
  Democrat &  0.37 &  0 &  1 &  0.48 \\ 
  Republican &  0.25 &  0 &  1 &  0.43 \\ 
  Conservative &  0.39 &  0 &  1 &  0.49 \\
\hline 
  \end{tabular}
}  
\end{center}
\end{frame}

\begin{frame}{Science Literacy Measure}

\begin{itemize}
\item \textbf{GHG}: \textit{What gas do most scientists believe causes
    temperatures in the atmosphere to rise?}

\item \textbf{Mars}: \textit{What have scientists recently discovered
    on Mars?}

\item \textbf{Aspirin}: \textit{Which over-the-counter drug do doctors
    recommend that people take to help prevent heart attacks?}

\item \textbf{Stem Cells}: \textit{How are stem cells different from
    other cells?}

\item \textbf{Evolution}: \textit{From what you’ve heard or read, do
    scientists generally agree that humans evolved over time, or do
    they not generally agree about this?}

\end{itemize}
  
\end{frame}

\begin{frame}{Science Literacy Summary Statistics}
\begin{center}
{
\begin{tabular}{llrrr}
 \textbf{Variable} & \textbf{Levels} & $\mathbf{n}$ & $\mathbf{\%}$ & $\mathbf{\sum \%}$ \\ 
  \hline
GHG & Incorrect & 643 & 32.1 & 32.1 \\ 
   & Correct & 1358 & 67.9 & 100.0 \\ 
   \hline
 & all & 2001 & 100.0 &  \\ 
   \hline
\hline
Mars & Incorrect & 750 & 37.5 & 37.5 \\ 
   & Correct & 1251 & 62.5 & 100.0 \\ 
   \hline
 & all & 2001 & 100.0 &  \\ 
   \hline
\hline
Aspirin & Incorrect & 150 & 7.5 & 7.5 \\ 
   & Correct & 1851 & 92.5 & 100.0 \\ 
   \hline
 & all & 2001 & 100.0 &  \\ 
   \hline
\hline
Stem Cells & Incorrect & 860 & 43.0 & 43.0 \\ 
   & Correct & 1141 & 57.0 & 100.0 \\ 
   \hline
 & all & 2001 & 100.0 &  \\ 
   \hline
\hline
Evolution & Incorrect & 583 & 33.1 & 33.1 \\ 
   & Correct & 1177 & 66.9 & 100.0 \\ 
   \hline
 & all & 1760 & 100.0 &  \\ 
   \hline
\hline
\end{tabular}
}
\end{center}
  
\end{frame}

\begin{frame}{Policy Areas}
\begin{itemize}
\item \textbf{Climate Change}: \textit{In your view, is global warming
    a very serious problem, somewhat serious, not too serious, or not
    a problem?}: 0 $=$ not too serious/not a problem; 1 $=$
  very/somewhat serious.
  
\item \textbf{Stem Cells}: \textit{All in all, do you favor or oppose
    federal funding for embryonic stem cell research?}: 0 $=$ Oppose;
  1 $=$ Favor

\item \textbf{Nuclear Energy}: \textit{All in all, do you favor or
    oppose building more nuclear power plants to generate
    electricity?}: 0 $=$ Oppose; 1 $=$ Favor 

\item \textbf{Vaccines}: \textit{Thinking about childhood diseases,
    such as measles, mumps, rubella and polio...}: 0 $=$ Should
  parents be able to decide NOT to vaccinate their children; 1 $=$
  Should all children be required to be vaccinated 

\end{itemize}
\end{frame}

\begin{frame}{Results}
\begin{center}
\includegraphics[width=3.25in]{all.pdf}
\end{center}  
\end{frame}

\begin{frame}{Results}
\includegraphics[width=\linewidth]{all2.pdf} 
\end{frame}


\begin{frame}{Conclusion}
  \begin{itemize}
  \item Despite some claims, particularly in the media and among
    political advocates, an increase in science literacy does not
    appear to reduce differences in opinion between conservatives and
    liberals over highly contentious scientific policy issues. Indeed,
    and consistent with research on polarization and sophistication,
    increased science literacy tends to exacerbate differences on
    these issues. This is likely because those that are knowledgeable
    are more “tuned-in” to the debate and better able to discern the
    positions of those with which they are affiliated. This
    polarization holds even when considering only information about
    the specific issue in question. 
  \end{itemize}
\end{frame}


\end{document}
