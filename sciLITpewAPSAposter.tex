\documentclass[a0,final]{a0poster}
\usepackage{multicol,graphicx,url,graphics}
\usepackage[font=small,labelfont=bf]{caption}
\usepackage[round]{natbib}
\bibpunct{(}{)}{;}{a}{}{,}
\usepackage[colorlinks=true,citecolor=blue]{hyperref}
\usepackage[left=2cm,right=2cm,bottom=0cm,top=0cm]{geometry}
\setlength{\columnsep}{2cm}
\setlength{\columnseprule}{1pt}
\makeatletter
\renewcommand\@maketitle{%
\null
{\Huge
\@title \par}%
\vskip 0.2em%
{\large
\lineskip .2em%
\begin{tabular}[t]{l}%
\@author
\end{tabular}\par}%
\vskip 1cm
\par}
\makeatother
\graphicspath{{./figures/}}

\title{Ideology, Science Literacy, and the Polarization of Science-Based Policy Issues}
\author{Matthew C. Nowlin, \textit{College of Charleston} \\
\url{http://www.matthewcnowlin.com}}


\begin{document}

\begin{minipage}{\textwidth}
\maketitle
\end{minipage}
\vspace{1cm}

\begin{multicols}{3}
\raggedcolumns

\section*{Research Question(s)}
\begin{itemize}
\item Does increased science literacy decrease polarization on science--based policy issues? 
\item How is science literacy moderated by ideological and partisan attachments 
\item Is there a difference in impacts based on \textit{general} scientific knowledge vs. \textit{issue specific} knowledge? 
\end{itemize}

\section*{Science Literacy and Public Opinion}

\noindent \textbf{Science Literacy} \\

\noindent Some argue that polarization on science issues result from a lack of basic understanding of science, or from hostility to science from certain groups.  Indeed, previous has some, on issue like climate change, knowledge tends to lead toward convergence of opinion 

\begin{center}
  \textit{H1: Increased science literacy will decrease issue polarization}
\end{center}

%\vspace{0.25in}
\noindent \textbf{Motivated Reasoning and Sophistication} \\

\noindent However, others argue that scientific information is processed in a “motivated” way, where information that confirms previous bias is accepted and information that is counter to bias is rejected. Going further, there is evidence to suggest that those with higher levels of sophistication (e.g., education, political knowledge, science literacy) tend to be those most polarized.   

\begin{center}
  \textit{H2: Increased science literacy will increase issue polarization}
\end{center}

\section*{Data and Methods}

The data used to test these hypotheses are from a \textit{Pew Research Center for the People and the Press} survey administered from April 28 to May 12, 2009. The RDD telephone survey was given to a sample of 2,001 adults in the U.S. through both landlines and cell phones.   

\subsubsection*{Independent Variables}

\begin{center}
{
\tiny
\begin{tabular}{lrrrr}
 \textbf{Variable} & $\mathbf{\bar{x}}$ & \textbf{Min} & \textbf{Max} & $\mathbf{s}$ \\ 
  \hline
Age & 51.37 & 18 & 95 & 17.58 \\ 
  Male &  0.50 &  0 &  1 &  0.50 \\ 
  White &  0.79 &  0 &  1 &  0.41 \\ 
  Education &  4.62 &  1 &  7 &  1.69 \\ 
  Income &  5.21 &  1 &  9 &  2.36 \\ 
  Born Again &  0.44 &  0 &  1 &  0.50 \\ 
  Church Attendance &  2.68 &  0 &  5 &  1.59 \\ 
  Liberal &  0.21 &  0 &  1 &  0.40 \\ 
  Democrat &  0.37 &  0 &  1 &  0.48 \\ 
  Republican &  0.25 &  0 &  1 &  0.43 \\ 
  Conservative &  0.39 &  0 &  1 &  0.49 \\
\hline 
  \end{tabular}
}
\end{center}

\subsubsection*{Science Literacy Measure}

The following five questions were asked on the \textit{Pew} survey. This is a smaller number of questions than what is typically used to measure science literacy, however the questions load on one factor using principal components analysis. 

\begin{itemize}
\item \textbf{GHG}: \textit{What gas do most scientists believe causes temperatures in the atmosphere to rise?}
\item \textbf{Mars}: \textit{What have scientists recently discovered on Mars?}
\item \textbf{Aspirin}: \textit{Which over-the-counter drug do doctors recommend that people take to help prevent heart attacks?}
\item \textbf{Stem Cells}: \textit{How are stem cells different from other cells?}
\item \textbf{Evolution}: \textit{From what you’ve heard or read, do scientists generally agree that humans evolved over time, or do they not generally agree about this?}
\end{itemize}

\vspace{0.25in}

\textbf{Science Literacy Summary Statistics} \\

\begin{center}
{
\begin{tabular}{llrrr}
\tiny
 \textbf{Variable} & \textbf{Levels} & $\mathbf{n}$ & $\mathbf{\%}$ & $\mathbf{\sum \%}$ \\ 
  \hline
GHG & Incorrect & 643 & 32.1 & 32.1 \\ 
   & Correct & 1358 & 67.9 & 100.0 \\ 
   \hline
 & all & 2001 & 100.0 &  \\ 
   \hline
\hline
Mars & Incorrect & 750 & 37.5 & 37.5 \\ 
   & Correct & 1251 & 62.5 & 100.0 \\ 
   \hline
 & all & 2001 & 100.0 &  \\ 
   \hline
\hline
Aspirin & Incorrect & 150 & 7.5 & 7.5 \\ 
   & Correct & 1851 & 92.5 & 100.0 \\ 
   \hline
 & all & 2001 & 100.0 &  \\ 
   \hline
\hline
Stem Cells & Incorrect & 860 & 43.0 & 43.0 \\ 
   & Correct & 1141 & 57.0 & 100.0 \\ 
   \hline
 & all & 2001 & 100.0 &  \\ 
   \hline
\hline
Evolution & Incorrect & 583 & 33.1 & 33.1 \\ 
   & Correct & 1177 & 66.9 & 100.0 \\ 
   \hline
 & all & 1760 & 100.0 &  \\ 
   \hline
\hline
\end{tabular}
}
\end{center}

\columnbreak

\subsection*{Policy Areas}

I exam the relationship between science literacy and polarization across four policy areas that are based on science and/or scientific research and findings. The following questions constituted the dependent variables. They were coded as 0 or 1 and logit analysis--with interaction terms for conservative and science literacy--was performed. 

\begin{itemize}
\item \textbf{Climate Change}: \textit{In your view, is global warming a very serious problem, somewhat serious, not too serious, or not a problem?}: 0 $=$ not too serious/not a problem; 1 $=$ very/somewhat serious.  
\item \textbf{Stem Cells}: \textit{All in all, do you favor or oppose federal funding for embryonic stem cell research?}: 0 $=$ Oppose; 1 $=$ Favor
\item \textbf{Nuclear Energy}: \textit{All in all, do you favor or oppose building more nuclear power plants to generate electricity?}: 0 $=$ Oppose; 1 $=$ Favor 
\item \textbf{Vaccines}: \textit{Thinking about childhood diseases, such as measles, mumps, rubella and polio...}: 0 $=$ Should parents be able to decide NOT to vaccinate their children; 1 $=$ Should all children be required to be vaccinated 
\end{itemize}

\section*{Results}

\captionof{figure}{\textit{Science Literacy and Polarization}}
\includegraphics[width=\linewidth]{all.pdf}

\subsection*{Science Literacy Interactions}

\begin{center}
\begin{tabular}{ l | c | c | c | c } 
\hline 
  & \multicolumn{ 1 }{ c| }{Climate Change} & \multicolumn{ 1 }{ c| }{Stem Cells} & \multicolumn{ 1 }{ c| }{Nuclear Energy} & \multicolumn{ 1 }{ c }{Vaccines} \\ \hline
Conservative * Science Literacy       & -0.27^*         & -0.54^{***}     & 0.18             & 0.07            \\ 
                 & (0.13)           & (0.12)           & (0.12)           & (0.12)           \\
\hline
\multicolumn{5}{l}{\footnotesize{Standard errors in parentheses}}\\
\multicolumn{5}{l}{\footnotesize{$^\dagger$ significant at $p<.10$; $^* p<.05$; $^{**} p<.01$; $^{***} p<.001$}} 
\end{tabular} 
\end{center}

\columnbreak


\captionof{figure}{\textit{Issue Specific Knowledge and Polarization}}
\includegraphics[width=\linewidth]{all2.pdf}


\subsection*{Issue Specific Interactions}
\begin{center}
\begin{tabular}{ l | c | c } 
\hline 
  & \multicolumn{ 1 }{ c |}{Climate Change} & \multicolumn{ 1 }{ c }{ Stem Cells } \\ \hline
Conservative * GHG        & -0.69^*     &             \\ 
               & (0.32)       &             \\ 
Conservative * Stem Cells &              & -0.62^*    \\ 
               &              & (0.28)       \\
\hline
\multicolumn{3}{l}{\footnotesize{Standard errors in parentheses}}\\
\multicolumn{3}{l}{\footnotesize{$^\dagger$ significant at $p<.10$; $^* p<.05$; $^{**} p<.01$; $^{***} p<.001$}} 
\end{tabular} 
\end{center}

\section*{Conclusion}

\noindent Despite some claims, particularly in the media and among political advocates, an increase in science literacy does not appear to reduce differences in opinion between conservatives and liberals over highly contentious scientific policy issues. Indeed, and consistent with research on polarization and sophistication, increased science literacy tends to exacerbate differences on these issues. This is likely because those that are knowledgeable are more “tuned-in” to the debate and better able to discern the positions of those with which they are affiliated. This polarization holds even when considering only information about the specific issue in question.   

\end{multicols}


\end{document}


